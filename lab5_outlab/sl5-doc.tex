\documentclass[12pt,a4paper]{article}
\usepackage[utf8]{inputenc}
\usepackage{bera}
\usepackage{geometry}
\geometry{a4paper, portrait, margin=1.1in}
\usepackage{graphicx}
\usepackage{amsmath}
\usepackage[noeepic]{qtree}
\usepackage{todonotes}
\usepackage[shortlabels]{enumitem}
\usepackage{framed}
\usepackage{epigraph}
\usepackage{algorithmic}
\usepackage{algorithm2e}
\usepackage{url}
\usepackage{bm}
\usepackage{listings}
\usepackage{xcolor}
\renewcommand{\labelitemi}{$\bullet$}
\newcommand{\norm}[1]{\left\lVert#1\right\rVert}
\lstdefinestyle{color}{
  basicstyle=\color{black},
  moredelim=**[is][\color{blue}]{#}{#},
  moredelim=**[is][\color{green}]{@ }{@}
}
\begin{document}


\begin{titlepage} 
    \centering     
    \vspace*{3.5cm}         
    
    \vspace*{2.4cm}     
 
    \Huge \textbf{Software Lab: OutLab\\Advanced \LaTeX}     
 
    \vspace*{0.8cm}     
 
    \LARGE{Pm9}
     
    \large{19305R009} \\
    \large{193050055}
     
    \large{193050023}
    \vspace*{0.5cm}     
 
    \large{28st August, 2019}     
    \vspace*{4.0cm}     
 
    \vspace*{\fill} 
\end{titlepage}
\setcounter{page}{1} 
\tableofcontents 

\newpage

You need to replicate the document

\section{Do this}


\begin{itemize}

\item \LaTeX{} typesets a file of text using the TEX program.
\\
\item \LaTeX{} is widely used in academia for the communication and publication
of scientific documents in many fields, including mathematics, physics,
computer science, statistics, economics and political science.
\\
\item \LaTeX{} can be used as a standalone document preparation system or as an
intermediate format.
\\
\item \textbf{Have used renewcommand for the bullets to be bigger.}
\\
\item look at the item separation space, and change it accordingly

\end{itemize}

\begin{enumerate}[I]

\item \LaTeX{} typesets a file of text using the TEX program.

\item \LaTeX{} is widely used in academia for the communication and publication
of scientific documents in many fields, including mathematics, physics,
computer science, statistics, economics and political science.

\item \LaTeX{} can be used as a standalone document preparation system or as an
intermediate format.

\item \LaTeX{} is intended to provide a high-level language that accesses the power
of TeX in an easier way for writers.

\end{enumerate}

\begin{enumerate}[label=(\alph*)]

\item \LaTeX{} typesets a file of text using the TEX program.

\item \LaTeX{} is widely used in academia for the communication and publication
of scientific documents in many fields, including mathematics, physics,
computer science, statistics, economics and political science.

\end{enumerate}

\newpage

\section{Mathematical formulas and notations}

\subsection{Equation Array}
     \begin{align}
     \cos^3{\theta} + \sin^3{\theta} &= (\cos{\theta} + \sin{\theta})(\cos^2{\theta} - \cos{\theta}\sin{\theta}) \\
    &= (\cos{\theta} + \sin{\theta})(1 - \cos{\theta}\sin{\theta}) \\
    &= (\cos{\theta} + \sin{\theta})(1 / 2)(2 - 2\cos{\theta}\sin{\theta}) \\
    &= (1 / 2)(\cos{\theta} + \sin{\theta})(2 - \sin{(2\theta)})
     \end{align}
 

\subsection{Prepositional Formulae using Various Operators}

 
\vspace*{0.5cm}
\hspace*{0.5cm}
$(\exists)(\boldsymbol{\varphi}(x)\wedge\boldsymbol{\psi}(x))\leftrightarrow((\exists x)\boldsymbol{\varphi}(x)\wedge(\exists x)\boldsymbol{\psi}(x))$
\vspace*{0.5cm}
\hspace*{0cm}

$((\forall x)\boldsymbol{\varphi}(x)\wedge(\forall x)\boldsymbol{\varphi}(x))\rightarrow(\forall x)(\boldsymbol{\psi}(x)\wedge\boldsymbol{\psi}(x)))$
\vspace*{0.5cm}
\hspace*{0cm}

${(\exists)(\boldsymbol{\varphi}(x)\wedge\boldsymbol{\psi}(x))\rightarrow((\exists x)\boldsymbol{\varphi}(x)\wedge(\exists x)\boldsymbol{\varphi}(x)\wedge(\exists x)\boldsymbol{\psi}(x)) }$

 
\subsection{Alphabets}
\vspace*{0.5cm}
\begin{center}
   \begin{tabular}{|c|c|}
	\hline
     Binary Operators & $\times \otimes \oplus \cup \cap$ \\
	& \\
	\hline
     Relation Operators & $\subset \supset \subseteq \supseteq < >$ \\
	& \\	
	\hline
     Others: & $\int \oint \Sigma \Pi$ \\
	& \\	
	\hline
   \end{tabular}
\end{center}
\vspace*{0.5cm}
\subsection{Mathematical Formulae}
     \begin{enumerate}
     %\begin{flalign*}% left aligned
     \item $\displaystyle\int_{a}^{b} x^3 dx = \frac{1}{4}x^4\bigg|_{a}^{b}$
     
     \item $\displaystyle\frac{\pi}{4} = 4 \sum_{n=1}^{\infty}\frac{(-1)^n}{(2n + 1)5^{2n+1}} - \sum_{n=0}^{\infty}\frac{(-1)^n}{(2n + 1)239^{2n+1}}$
     
     \item $\displaystyle\pi = \frac{3\sqrt{3}}{4} - 24 \sum_{n=0}^{\infty}\frac{\frac{(2n)!}{(n)}}{2n + 1(2n + 1)4^{2n+1}}$
     
    \newpage
     
     \item $\displaystyle \frac{1}{\pi} = \frac{2\sqrt{2}}{9801}\sum_{n=0}^{\infty}\frac{(4n)!(1103 + 26390n)}{(n)!^{4}396^{4n}}$
     
     \item $\sum_{i=1}^{[\frac{n}{2}]}$ $\displaystyle{{x_{i,i+1}^{i^2}}\choose [\frac{i+3}{3}]} \frac{\sqrt{\mu(i)^{\frac{3}{2}} (i^2 - 1)}}{\sqrt[3]{\rho(i)-2} + \sqrt[3]{\rho(i)-1}}$
     
     \item $\lim_{{(v,v')}\to{(0,0)}}$ $\displaystyle\frac{H(z+v) - H(z+v') - BH(z)(v-v')}{\norm{v-v'}}$

     \item $det \textbf{K}(t=1,t_1,...,t_n) = \sum_{I\in n}(-1)^{|I|}\prod_{i\in I}t_i\prod_{j\in I}(D_j+\lambda_j t_j) det\textbf{A}^{(\lambda)}(\overline{I}|\overline{I}) = 0$
     %\end{flalign*}
     
     \end{enumerate}
     $\frac{4}{\pi^2}=\frac{4\sqrt{2}}{\pi}\sum\limits_{n=0}^{\infty} \frac{(2n)!}{6^{4n}}$

{

\section{Quotation and Citation}
\subsection{Quotation}
The margins of the quotation environment are indented on both the left and the
right. The text is justified at both margins. Leaving a blank line between text
produces a new paragraph. The package \textbf{csquotes} offers a multilingual solu-
tion to quotations, with integration to citation mechanisms offered by BibTeX.
This package allows one for example to switch languages and quotation styles
according to babel language selections.
\\

\setlength{\leftskip}{1cm}
\setlength{\rightskip}{1cm}

{“Unlike the quote environment, each paragraph is indented normally. It’s important to remark that even if you are typing quotes on English there are different quotation marks used in English (UK) and English (US).”}

\setlength{\leftskip}{0cm}
\setlength{\rightskip}{0cm}

\subsection{Citation}
Latex\cite{firuza_aibara} is a document preparation system for typesetting program. It is used
to create different types of document structures. A Latex file (.tex) is created
using any text editor (vim, emacs, gedit, etc.). There are also many LaTeX IDEs
like Kile, TexStudio, etc.. The Latex code is then compiled which creates a
standard (.pdf) file. Thus, the presentation of the document does not change on
different machines.

Type style\cite{latex_leslie} is used to indicate logical structure. Emphasized text appears in
italic style type and input in typewriter style. Type style is specified by three
components: shape, series, and family.

\newpage
There are two ways of producing a bibliography\cite{latex_hopka}. You can either produce
a bibliography by manually listing the entries of the bibliography or producing
it automatically using the BibTeX program of LaTeX. The bibliography style can
be declared with bibliographystyle command, which may be issued anywhere
after the preamble. The style is a file with .bst extension that determines how
bibliography entries will appear at the output, such as if they are sorted or not,
or how they are labeled etc. The extension .bib is not written explicitly. There
are many standard bibliography style files. Two of them that are compatible
with IIT thesis manual are plain.bst and alpha.bst. They are part of the LaTeX
package; a student does not need to download it. The plain.bst and alpha.bst
styles are explained below. The symbols in a math formula fall into different
classes that correspond more or less to the part of speech each symbol would
have if the formula were ex pressed in words. Certain spacing and position-
ing cues are traditionally used for the different symbol classes to increase the
readability of formulas. \cite{latex_michael}

My citations are in proper order as per references ref1, ref2, ref3, and ref4.

\newpage
\section{Algorithm and Pseudo Code}
\subsection{Listing}

\hrulefill
\large{
\begin{lstlisting}[style=color]
@// Breadth First Search Function@
#void# BFS(list<#long long int#> queue ,#long long int# length
    ){
     #long long int# v;
     #if#(queue.empty())
         #return#;
     list<#long long int#>::iterator i;
     list<#long long int#> queue_temp ;
     #while# (!queue.empty()){
         v=queue.front();
         queue.pop_front();
         #for#(i=adj[v].begin();i!=adj[v].end();i++){
             #if#(!pro_ver[*i]){
                 result[*i]=length;
                 queue_temp.push_back(*i);
                 pro_ver[*i]=#true#;
                 adj[*i].remove(v);
             }
         }
     }
     BFS(queue_temp,length+1) ;
}
\end{lstlisting}}
\hrulefill
\newpage

\subsection{Verbatim}
 \large{
\begin{verbatim}
//Breadth First Search Function
void BFS(list<long long int> queue,long long int length){
    long long int v;
    if(queue.empty())
        return;
    list<long long int>::iterator i;
    list<long long int> queue_temp;
    while(!queue.empty()){
        v=queue.front();
        queue.pop_front();

        for(i=adj[v].begin();i!=adj[v].end();i++){
            if(!pro_ver[*i]){
                result[*i]=length;
                queue_temp.push_back(*i);
                pro_ver[*i]=true;
                adj[*i].remove(v);
            }
        }
    }
    BFS(queue_temp,length+1);
}
\end{verbatim}
}
\clearpage



\subsection{Algorithmic}
{
\begin{algorithm}
\begin{framed}
\SetKwFunction{BFS}{Breadth-First-Search}
\SetKwInOut{Input}{Input}
\SetKwInOut{Output}{Output}
\SetKwProg{Func}{}{}{}

\Input{A graph Graph and a starting vertex root of Graph}
\Output{All vertices’s reachable from root labeled as explored.}
\Func {
    \BFS{Graph, root}:\\
    \For{each node n in Graph} {
        {n.\textbf{distance} = INFINITY} \\
        {n.\textbf{parent} = NIL}
    }
    {create empty \textbf{queue} Q} \\
    {root.\textbf{distance} = 0} \\
    {Q.\textbf{enqueue}(root)} \\
    \While{Q is not empty} {
    current = Q.dequeue()\\
    \For{each node n that is ad jacent to current}{
        \If{n.\textbf{distance} == INFINITY} {
            {n.\textbf{distance} = current.\textbf{distance} + 1} \\
            {n.\textbf{parent} = current} \\
            {Q.\textbf{enqueue}(n) }
        }
    }
}
}

\end{framed}

\caption{Breadth-First-Search}
\end{algorithm}
}}
\section{Tree}

\Tree[.pm9 [.Amit sed awk ] [.Akhil [.HTML css ] latex linux ] [.Shivam [.bash bashrc ] git ] ]
 
\textbf{\small Here, you need to Build your family tree apart from the tree shown.
The next part of the tree assignment requires you to build your family
tree upto level 3 i.e. if you are a leaf node at the third level, your parents
and their siblings are level 2, their parents are level 1. In case you do not know names, of someone in family tree, please assume. This will be
manually evaluated}


\section{Exotic Features}

\subsection{Epigraph Style}

{\textbf{Chapter 1: Theory of life}}

\epigraph{\textit{“failure will never overtake me if
my determination to succeed is
strong enough.”}}{\textit{og mandino}}

\subsection{Minipage}

\fbox{
\begin{minipage}{0.46\textwidth}
\textit{\small
\LaTeX{} typesets a file of text using the TEX program and the \LaTeX{} “macro package” for TEX. That is, it processes an input file containing the text of a document with interspersed commands that describe how the text should be formatted. \LaTeX{} files are plain text that can be written in any reasonable editor. In the
\LaTeX{} input file, a command name starts with a followed by either (a) a string of
letters or (b) a single non-letter. Arguments contained in square brackets, [],
are optional while arguments contained in braces, \{\}, are required. \LaTeX{} is case
sensitive. Enter all commands in lower case unless explicitly directed to do otherwise.
}
\end{minipage}
}




\clearpage
 \section{Bibliography}
 \bibliographystyle{unsrt}
 \bibliography{sl5-doc} 

\end{document}
