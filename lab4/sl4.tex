\documentclass[12pt, a4paper]{article}
\usepackage[utf8]{inputenc}
\usepackage{bera}

\newcommand*{\qcr}{\fontfamily{qcr}\selectfont}

\begin{document} 

\begin{titlepage}
\centering
    \vspace*{\fill}

    
    \Huge \textbf{Software Lab 4:\\TeX Lab}
    
    \vspace*{0.6cm}
    
    \LARGE{Diptesh Kanojia - Title Page}
    \vspace*{0.4cm}

    \Large{21st August, 2019}
    \vspace*{4.0cm}

    \vspace*{\fill}
\end{titlepage}

\setcounter{page}{1}
\tableofcontents

\clearpage

\section{About Me}
Hello, my name is Amit Jaiswal. You are probably wondering what
kind of a narcissist uses his own name as the title of a document - in
my defense, a lazy one! I am currently pursuing M.Tech. in Computer Science. This is a \LaTeX{} document for the course \textbf{Software Lab} with course code \emph{CS699}. I would like for this document to be typesetted perfectly which forces me to you use \LaTeX{}. \LaTeX{} uses various packages. I will elaborate about them in the following subsections:

\subsection{graphicx package}
This package is used to import tables, and figure in the document. Our
document type is article, and we are currently using 11pt font size, with
a4 type paper, which is specified in the beginning in $<$documentclass$>$.

\subsection{amssymb package}
This package is used to import mathematical symbols in the document.
We encapsulate the mathematical equations and symbols under \$, and
they are changed to maths symbols.

\section{Some History}
I am ancient creature dwelling on this planet now referred to as 'Earth'.
I have been existing since the past $150393894.5$ years. Do you see the
use of a package above in the number mention in the document. I
have used something to enunciate the numbers in a fashion such as a
mathematical formulae.

Let us all try to replicate the text provided in this document.

\textit{P.S.: Please note that I am following the Section Title \textbf{Noun Capitalization} in the document. This would be followed in the rest of the
document, henceforth.}

\clearpage

\section{Replication of this must be produced}
\textbf{LaTeX} is a word processor and document markup language. It is distinguished from typical word processors such as Microsoft Word and
Apple Pages in that the writer uses plain text as opposed to formatted text, relying on markup tagging conventions to define the general
structure of a document (such as article, book, and letter), to stylise
text throughout a document (such as \textbf{bold} and \textit{italic}), and to add citations and 
crossreferencing. A \textbf{TeX} distribution such as \textbf{TeXlive} or
\textbf{MikTeX} is used to produce an output file (such as PDF or DVI) suitable
for printing or digital distribution.
\\*

\textbf{LaTeX} is used for the communication and publication of scientific
documents in many fields, including mathematics, physics, computer
science, statistics, economics, and political science. It also has a prominent role in the preparation and publication of books and articles that
contain complex multilingual materials, such as Sanskrit and Arabic. 
\textbf{LaTeX} uses the TeX typesetting program for formatting its output, and
is itself written in the TeX macro language.
\\*

\textbf{LaTeX} is widely used in academia. \textbf{LaTeX} can be used as a standalone document preparation system, or as an intermediate format. In
the latter role, for example, it is often used as part of a pipeline for
translating DocBook and other XML-based formats to PDF. The typesetting system offers programmable desktop publishing features and
extensive facilities for automating most aspects of typesetting and desktop publishing, including numbering and cross-referencing of tables
and figures, chapter and section headings, the inclusion of graphics,
page layout, indexing and bibliographies.
\\*

Like \textbf{TeX}, \textbf{LaTeX} started as a writing tool for mathematicians and
computer scientists, but from early in its development it has also been
taken up by scholars who needed to write documents that include complex math expressions or non-Latin scripts, such as Arabic, Sanskrit and Chinese.
\\*

\textbf{LaTeX} is intended to provide a high-level language that accesses the
power of \textbf{TeX}. \LaTeX{} comprises a collection of TeX macros and a program
to process \textbf{LaTeX} documents. Because the plain \textbf{TeX} formatting commands are elementary, it provides authors with ready-made commands for formatting and layout requirements such as chapter headings, foot-notes, cross-references and bibliographies.
\\*

\textbf{LaTeX} was originally written in the early 1980s by Leslie Lamport at
SRI International. The current version is LaTeX2e. \textbf{LaTeX} is free software and is distributed under the \textbf(LaTeX) Project Public License (LPPL)
\textbf{(Source Wikipedia)}.

\section{Opening and Compiling Tex Document}

First create a \textbf{.tex} file using text editor such as \textbf{Vi} or \textbf{Gedit} or \textbf{Kile}.

\section{Starting and Ending}

A minimal input file looks like following
\textbf{\qcr \large
\newline \newline
\hspace*{1cm}\textbackslash documentclass\{class\}
\\*
\hspace*{1cm}\textbackslash \{document\}
\\*
\hspace*{1.8cm}your text...
\\*
\hspace*{1cm}\textbackslash end \{document\}
}
\newline \newline
where the class is a valid document class for \textbf{LaTeX}.

\subsection{Compiling the LaTeX Document}

We open the terminal and go to the directory in which our .tex file is
stored and the we execute the command\\

\textbf{pdflatex example.tex}

\clearpage

\section{Section}
Sectioning commands provide the means to structure your text into
\\*
units:
\\~\\
\newline
{\qcr \large \textbackslash{}\textbf{part}
\\~\\
\textbackslash{}\textbf{chapter}
\\~\\
\textbf{(report and book class only)}
\\~\\
\textbackslash{}\textbf{section}
\\~\\
\textbackslash{}\textbf{subsection}
\\~\\
\textbackslash{}\textbf{subsubsection}
\\~\\
\textbackslash{}\textbf{paragraph}
\\~\\
\textbackslash{}\textbf{subparagraph}\\
}
\\*
\hspace*{0.5cm}All sectioning commands take the same general form, e.g.,
\\~\\
\hspace*{2cm}\textbackslash{}\textbf{\qcr \large chapter[toctitle]\{title\}}
\\~\\
\hspace*{0.5cm}In addition to providing the heading title in the main text, the section
\\*
title can appear in two other places:
\begin{enumerate}
  \item The table of contents.
  \item The running head at the top of the page.
\end{enumerate}
\hspace{0.5cm}You may not want the same text in these places as in the main text.
To handle this, the sectioning commands have an optional argument
toctitle that, when given, specifies the text for these other places.
\\~\\
\hspace{0.5cm}Also, all sectioning commands have *-forms that print title as usual,
butdo not include a number and do not make an entry in the table of
contents.
\\~\\
\hspace*{0.5cm}For instance:
\\~\\
\hspace*{2cm}\textbackslash{}\textbf{\qcr \large section*\{Preamble\}}
\\~\\
\hspace*{0.5cm}The \textbackslash{}\textbf{appendix} command changes the way following sectional units
are numbered. The \textbackslash{}\textbf{appendix} command itself generates no text and
does not affect the numbering of parts.
\\~\\
\hspace*{0.5cm}The normal use of this command is something like
\\~\\
{\qcr \large
\hspace*{2cm}\textbackslash{}\textbf{chapter\{A Chapter\}}
\\
\hspace*{2cm}...
\\
\hspace*{2cm}\textbackslash{}\textbf{appendix}
\\
\hspace*{2cm}\textbackslash{}\textbf{chapter\{The First Appendix\}}
}
\\~\\
\hspace{0.5cm}The secnumdepth counter controls printing of section numbers. The
setting suppresses heading numbers at any depth $>$ level, where chapter is level zero.
\\~\\
\hspace*{2cm}\textbackslash{}\textbf{\qcr \large setcounter\{secnumdepth\}\{level\}}

\section{Cross Reference}
One reason for numbering things like figures and equations is to refer
the reader to them, as in “Section 6 on Page 5 for more details”. {\Large I
have referred to a section, and a page here here.}
\subsection{\textbackslash label\{key\}}

A \textbackslash label command appearing in ordinary text assigns to key the number
of the current sectional unit; one appearing inside a numbered environ-
ment assigns that number to key.
\par
A key name can consist of any sequence of letters, digits, or punctu-
ation characters. Upper and lowercase letters are distinguished.
\par
To avoid accidentally creating two labels with the same name, it is
common to use labels consisting of a prefix and a suffix separated by a
colon or period. Some conventionally-used prefixes:
\newline
{\bf ch}\qquad for chapters \newline
{\bf sec}\qquad for lower-level sectioning commands \newline
{\bf fig}\qquad for figures \newline
{\bf tab}\qquad for tables \newline
{\bf eq}\qquad for equations \newline

{\Huge I think we have replicated the document enough. Let us just concentrate on learning features of the
document provided to us. We have
successfully demonstrated the the
features such as Sections, Subsec-
tions, Labelling, Bold, Italics, Tab-
bing, Title Page, Huge, Large, math
symbols. Typing in a \LaTeX{} document
to type in \textbf{LaTeX} code.}
\newline
\par Let us leave the rest of \LaTeX{} for the out-lab.

\end{document}